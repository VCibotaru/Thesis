% \chapter{Аннотация}\label{Abstract}

\setlength{\parskip}{0pt}
\begin{center}
	{\LARGE{\textbf{Аннотация}} \par}
\end{center}
\bigskip
     В данной работе рассматривается метод автоматического анализа веб-приложений с целью выявления способов эксплуатации уязвимостей типа DOM Clobbering, связанных с возможностью изменения области видимости переменных объектной модели веб-страницы. Идея метода состоит в отслеживании и дальнейшем анализе потоков данных в коде веб-страницы, исполняемого на стороне клиента.
     В работе приводится обзор существующих методов анализа кода, написанного на языке JavaScript и предлагается использование метода тейнтирования (taint analysis).
     Предложенные идеи реализованы в виде дополнения к браузеру Mozilla Firefox. Для проверки корректности его работы было проведено тестирование как на синтетических наборах тестов, так и на реальных веб-приложениях.
\bigskip



