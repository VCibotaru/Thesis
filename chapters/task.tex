% Task

\chapter{Постановка задачи}\label{Introduction}

\section{Неформальная постановка задачи}

Рассмотрим на небольшом примере причины возникновения недостатка DOM Clobbering.

\bigskip
% TODO пронумеровать листинги
\textbf{Пример 1:}

\begin{lstlisting}
	<form name="form_name">
	<script>
		var form = document.form_name; // указывает на <form name="form_name">
	</script>
\end{lstlisting}
\bigskip




Для представления содержимого веб-страницы в виде объектов на языке JavaScript веб-браузеры используют интерфейс DOM (Объектная Модель Документа). В рамках DOM HTML-странице ставится в соответствие объект document, а окну веб-браузера - объект window. После загрузки и обработки страницы эти объекты заполняются различным свойствами (например, document.location - объект, содержащий информацию про расположение (URL) документа).


% TODO: ссылка на Tangled Web
Так же дело обстоит и с тэгами страницы: после создания HTML элемента <x name='element\_name' id='element\_id'> он становится доступен по указателям document.element\_name, window.element\_name (верно для элементов одного из типов <img>, <form>, <embed>, <object> и <applet>) и window.element\_id (верно для всех элементов), где element\_name - это имя (атрибут name) созданного элемента, а element\_id - это идентификатор элемента (атрибут id).



Аналогично в пространство имен документа попадают и объекты из JavaScript кода. Например, в результате загрузки страницы в пространство имен документа добавится выше описанное свойство document.location. То есть, свойства DOM могут заполняться как из контекста JavaScript кода, так и в результате обработки HTML элементов.


Такое поведение приводит к тому, что если в пространстве имен документа уже существовал объект document.x и HTML парсер встречает элемент <form name='x'>, то происходит конфликт имен и document.x начинает указывать на HTML форму. В этом заключается вся суть эксплуатации недостатка DOM Clobbering: злоумышленник подбирает имя HTML элемента таким образом, чтобы заменить какой-нибудь важный объект в объектной модели документа страницы.


Рассмотрим пару типичных примеров недостатка DOM Clobbering.

\bigskip
\textbf{Пример 2:}


\begin{lstlisting}
	<form name="querySelector">
	<script>
		var element = document.querySelector("a"); // ошибка: document.querySelector указывает на <form name=“querySelector”
	</script>
\end{lstlisting}
\bigskip

В данном примере пользователь путем взаимодействия с приложением может изменять значение атрибута name (имя) у формы на странице и устанавливает его равным “querySelector”. Далее, в коде вызывается вызывается функция document.querySelector, но, так как document.querySelector теперь указывает на HTML форму, которая не является функцией, при таком вызове произойдет ошибка.
То есть, действуя подобным образом, злоумышленник может добиться нарушения работоспособности кода на стороне клиента.

\bigskip
\textbf{Пример 3:}

\begin{lstlisting}
	<form name="is_in_black_list">
	<script>
		if (user_in_black_list()) {
			is_in_black_list = true;
		}
		if (is_in_black_list) {
			forbid_action();
		}
	</script>
\end{lstlisting}
\bigskip

В данном примере пользователь путем взаимодействия с приложением может изменять значение атрибута name (имя) у формы на странице и устанавливает его равным “is\_in\_black\_list". Далее вызывается функция \\user\_in\_black\_list, суть которой заключается в проверке находится ли пользователь в черном списке приложения. Если это так, то глобальной переменной is\_in\_black\_list присваивается значение true, иначе она остается неинициализированной. Далее, в зависимости от значения переменной is\_in\_black\_list пользователю разрешается или запрещается выполнение какого-нибудь действия. Но, так как имя формы  “is\_in\_black\_list”, то во втором условном операторе is\_in\_black\_list будет указывать на HTML форму, следовательно, пользователю будет отказано в выполнение действия.
Заметим, что данный пример является несколько вырожденным, но он ясно дает понять, что в некоторых случаях злоумышленники могут обойти логику работы кода на стороне клиента с помощью недостатка DOM Clobbering и тем самым навредить пользователям.

\bigskip
\textbf{Пример 4:}

\begin{lstlisting}
	<a href="plugins/preview/preview.html#<svg onload=alert(1)>" id="_cke_htmlToLoad" target="_blank">
		Click me!
	</a>

	файл /plugins/preview/preview.html:
	<script>
		...
		document.write(window.opener._cke_htmlToLoad);
		...
	</script>
\end{lstlisting}
\bigskip


% TODO: ссылка на спич Mario
Данный пример взят из работы реального веб-приложения. На одной из его страниц размещалась ссылка (элемент <a>), с идентификатором равным \\
“\_cke\_htmlToLoad”, указывающая на страницу “plugins/preview/preview.html\#<svg onload=alert(1)>”. Далее, после того, как пользователь переходил по этой ссылке, на странице plugins/preview/preview.html отрабатывал код, записывающий строку window.opener.\_cke\_htmlToLoad в конец веб-страницы. Но, так как window.opener указывал на ту страницу, с которой был осуществлен переход на текущую, а в ней \_cke\_htmlToLoad указывал на контролируемый элемент <a>, в конец документа записывалась строка “plugins/preview/preview.html\#<svg onload=alert(1)>”. А дописывание в документ строки <svg onload=alert(1)> означало создание HTML элемента типа <svg>, при завершении загрузки которого выполнялся код alert(1).
Таким образом, злоумышленник получал возможность внедрять и исполнять произвольный код на стороне клиента, что может привести к плохим последствиям.

\section{Формальное определение DOM Clobbering}

Дадим формальное определение DOM Clobbering: DOM Clobbering - это недостаток клиентской стороны веб-приложения, заключающийся в возможности подмены объектов (переменных) веб-страницы с помощью изменения имен и/или идентификаторов некоторых HTML элементов на веб-странице.

\section{Постановка задачи}
Основная задача данной работы - сформулировать методику и разработать инструментальное средство для определения возможности (и последствий) подмены объектов DOM для заданной веб-страницы.