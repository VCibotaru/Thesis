% Introduction

\chapter{Введение}\label{Introduction}

В период зарождения всемирной паутины веб-страницы представляли собой статические документы, сверстанные на языке HMTL. Но, с течением времени, страницы становились все более и более динамическими. Сегодняшние приложения обычно содержат большое количество кода, исполняемого на стороне клиента (в веб-браузере). Стоит отметить, что несмотря на существование конкурентов, язык JavaScript используется в подавляющем большинстве веб-приложений для написания кода для клиентских частей. Об этом факте свидетельствует бурный рост популярности данного языка и технологий, которые его используют (например, NodeJS \cite{nodejs}, AngularJS \cite{angularjs} и т. д.)


% TODO: надо ссылку на SPA
Активное использование языка JavaScript привело к появлению новой парадигмы устройства веб-приложений, называемой Single Page Application (Одностраничное приложение) \cite{spa}. Работая с таким приложением, пользователь все время находится в рамках одной веб-страницы, в коде которой и реализована большая часть бизнес логики. Всё общение между клиентом и веб-сервером осуществляется с помощью AJAX \cite{ajax} запросов без необходимости перезагрузки или перехода на другие страницы.


Высокая популярность Single Page Application наглядно демонстрирует тенденцию к всё большему усложнению кода клиентской части. Однако, хорошо известно, что рост сложности системы часто ведёт к снижению её безопасности. В случае веб-приложений этот факт доказывается большим количеством недостатков на стороне клиента. Вот самые опасные и распространенные из них:

\begin{enumerate}
	\item DOM Based XSS \cite{domxss}.
	\item DOM Redirection \cite{domredirect}.
	\item Некорректное использование механизмов Same Origin Policy \cite{sop}.
	\item DOM Clobbering.
\end{enumerate}


Самым распространенным из них является недостаток DOM Based XSS. В рейтинге недостатков веб-приложений OWASP (Open Web Application Security Project \cite{owasp}) Top Ten \cite{topten} он занимает третье место. Он заключается в том, что код веб-страницы обрабатывает пользовательские данные и модифицирует её содержимое, позволяя злоумышленнику исполнять произвольные команды. Этот тип недостатков достаточно хорошо исследован и для него были разработаны эффективные автоматические средства обнаружения. В виду некоторых особенностей языка JavaScript \cite{Richards2010} (слабая типизация, возможность динамического исполнения кода) статические методы анализа кода оказываются непригодными для решения подобных проблем. Поэтому разработанные решения опираются на такие методы, как динамический taint-анализ и фаззинг \cite{miller}.


Уязвимость, рассматриваемая в данной работе носит название DOM Clobbering (от англ. DOM - Document Object Model \cite{dom}, Объектная Модель Документа и Clobber - перезаписывать). Суть DOM Clobbering заключается в возможности подмены объектов (переменных) веб-страницы с помощью изменения их области видимости. Как и DOM-based XSS, DOM Clobbering - это недостаток целиком на стороне клиента. Стоит отметить, что в реальных сайтах DOM Clobbering встречается редко, что и является причиной его недостаточной изученности. Однако, высокая степень опасности, которую таят в себе недостатки данного типа, указывает на необходимость разработки метода и автоматизированного средства их поиска.