%Task Analysis

\chapter{Анализ задачи}\label{Analysis}

\section{Исследование возможных направлений атаки}

После исследования реальных примеров веб-страниц, обладающих недостатков DOM Clobbering, авторами были выявлены следующие возможные направления атаки.
\bigskip

\subsection{Вызов ошибки в программе}
% TODO: заменить ссылки на главы
Во-первых, злоумышленник может вызвать возникновение ошибки в коде клиентской стороны. Самый простой пример - это замена указателя на стандартную функцию, предоставляемую интерфейсом браузера, указателем на HTML элемент (пример 2 из главы Постановка задачи). Последствиями такой атаки является некорректная дальнейшая работа веб-приложения.

% \subsection{Сброс пользовательских настроек}
% Во втором направлении атаки злоумышленник также подменяет важные для работы приложения объекты, но на этот раз не с целью вызова ошибки в коде, а для того, чтобы испортить пользовательские настройки. Такой тип атаки особенно актуален для описанных выше приложений вида Single Page Application, в которых обычно результаты работы пользователя сохраняются в некий объект (назовем его state) для последующей отправки на сервер.

\subsection{Обход критических состояний}
% TODO: заменить ссылки на главы
Для описания второго направления, проанализируем пример 3 из главы Постановка задачи:

\begin{lstlisting}[language=HTML]
	<form name=“is_in_black_list”>
	<script>
		if (user_in_black_list()) { // точка 1
			is_in_black_list = true; // точка 2
		}
		if (is_in_black_list) { // точка 3
			forbid_action(); // точка 4
		}
	</script>
\end{lstlisting}
\bigskip

Предположим, что функция user\_in\_black\_list осуществляет посылку запроса на веб-сервер с целью выяснения находится ли пользователь в черном списке.


%TODO может быть вынести эти определения наверх?
Введем пару определений:
\begin{itemize}
	\item \textit{Состояние программы в точке $i$} - это тройка $<i, Vars, Values>$, где $i$ - точка программы, $Vars$ - множество переменных, доступных в $i$, $Values$ - множество двоек $<var, value>$, где $var \in Vars$, а $value$ - значение $var$ в $i$.  
	%TODO норм ли? критический - 2. Опасный, связанный с возможностью нарушения нормального состояния чего-нибудь
	\item \textit{Критическое состояние} - это такое состояние, в которое может перейти страница только при условии выполнения определенных требований на серверной стороне приложения.
	\item \textit{Некритическое состояние} - это состояние, не являющееся критическим.
	\item \textit{Допустимое для точки $i$ состояние} - это состояние $<i, Vars, Values>$, в котором программа может оказаться.
\end{itemize}


Например, состояние $<4, {is\_in\_black\_list} , {is\_in\_black\_list, true>}>$ является критическим, так как для того, чтобы программа перешла в него, необходимо присутствие пользователя в черном списке.

\bigskip
С другой стороны, состояние $<4, {is\_in\_black\_list}, {<is\_in\_black\_list, HTMLForm>}>$  также является допустимым для точки 4, но не является критическим.
\bigskip

Другими словами, для точки 4 существуют два допустимых состояния, одно из которых дает нарушителю возможность нанести вред пользователю, который не находится в черном списке. Очевидно, такая ситуация является угрозой безопасности приложения.


С помощью введенных выше определений второе направление атаки можно описать так: \textit{С помощью недостатка DOM Clobbering нарушителям удается перевести программу в некритическое состояние в некоторой точке, для которой все прочие допустимые состояния являются критическими}

\subsection{Выполнение произвольного кода}
% TODO: заменить ссылки на главы
В-третьих, злоумышленники могут добиться исполнения произвольного кода в веб-браузере жертвы. Реальным примером такой атаки является пример 4 из главы Постановка задачи. Стоит отметить, что в данном случае методы эксплуатации идентичны методам эксплуатации недостатка DOM Based XSS. Данная атака является самой опасной из приведенных трех, так как она позволяет нарушителям осуществлять любые действия от имени жертвы.

\section{Требования к задаче}
На основе проведенного исследования возможных направлений атаки были сформулированны следующие требование к алгоритму анализа веб-приложений.

\begin{enumerate}
	\item Алгоритм должен уметь детектировать возможность возникновения ошибок из-за переименования HTML элемента и места таких ошибок.
	%TODO уточнить формулировку
	\item Алгоритм должен уметь детектировать возможность и пути обхода критических состояний.
	\item Алгоритм должен уметь детектировать возможность выполнения произвольного кода.
\end{enumerate}

