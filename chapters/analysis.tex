%Task Analysis

\chapter{Анализ задачи}\label{Analysis}

\section{Исследование возможных направлений атаки}

После исследования реальных примеров веб-страниц, обладающих недостатков DOM Clobbering, авторами были выявлены три возможных направления атаки.
\bigskip

\subsection{Вызов ошибки в программе}
% TODO: заменить ссылки на главы
Во-первых, злоумышленник может вызвать возникновение ошибки в коде клиентской стороны. Самый простой пример - это замена указателя на стандартную функцию, предоставляемую интерфейсом браузера, указателем на HTML элемент (пример 2 из главы Постановка задачи). Последствиями такой атаки является некорректная дальнейшая работа веб-приложения.

\subsection{Обход критических состояний}
% TODO: заменить ссылки на главы
Для описания второго направления, проанализируем пример 3 из главы Постановка задачи:

\begin{lstlisting}[language=HTML]
	<form name=“security_flag”>
	<script>
		if (securityCheck()) { // точка 1
			security_flag = true; // точка 2
		}
		if (security_flag) { // точка 3
			criticalFunction(); //точка 4
		}
	</script>
\end{lstlisting}
\bigskip

Предположим, что функция securityCheck() осуществляет посылку запроса на веб-сервер с целью выяснения прав текущего пользователя на выполнение функции criticalFunction(). Будем считать, что данную проверку могут пройти только определенные пользователи (назовем их администраторами), и что злоумышленник не принадлежит к их числу.


%TODO может быть вынести эти определения наверх?
Введем пару определений:
\begin{itemize}
	\item \textit{Состояние программы в точке $i$} - это тройка $<i, Vars, Values>$, где $i$ - точка программы, $Vars$ - множество переменных, доступных в $i$, $Values$ - множество двоек $<var, value>$, где $var \in Vars$, а $value$ - значение $var$ в $i$.  
	%TODO норм ли? критический - 2. Опасный, связанный с возможностью нарушения нормального состояния чего-нибудь
	\item \textit{Критическое состояние} - это такое состояние, в которое может перейти страница только при условии выполнения определенных требований на серверной стороне приложения.
	\item \textit{Некритическое состояние} - это состояние, не являющееся критическим.
	\item \textit{Допустимое для точки $i$ состояние} - это состояние $<i, Vars, Values>$, в котором программа может оказаться.
\end{itemize}


Например, состояние $<4, {security\_flag}, {<security\_flag, true>}>$ является критическим, так как для того, чтобы программа перешла в него, необходима принадлежность пользователя к группе администраторов. 

\bigskip
С другой стороны, состояние $<4, {security\_flag}, {<security\_flag, HTMLForm>}>$  также является допустимым для точки 4, но не является критическим.
\bigskip

Другими словами, для точки 4 существуют два допустимых состояния, одно из которых дает нарушителю возможность совершения действий, на которые у него не хватает прав. Очевидно, такая ситуация является угрозой безопасности приложения.


С помощью введенных выше определений второе направление атаки можно описать так: \textit{С помощью недостатка DOM Clobbering нарушителям удается перевести программу в некритическое состояние в некоторой точке, для которой все прочие допустимые состояния являются критическими}

\subsection{Выполнение произвольного кода}
% TODO: заменить ссылки на главы
В-третьих, злоумышленники могут добиться исполнения произвольного кода в веб-браузере жертвы. Реальным примером такой атаки является пример 4 из главы Постановка задачи. Стоит отметить, что в данном случае методы эксплуатации идентичны методам эксплуатации недостатка DOM Based XSS. Данная атака является самой опасной из приведенных трех, так как она позволяет нарушителям осуществлять любые действия от имени жертвы.