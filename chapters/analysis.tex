%Task Analysis

\chapter{Анализ задачи}\label{Analysis}

\section{Исследование возможных направлений атаки}

После исследования реальных примеров веб-страниц, обладающих недостатком DOM Clobbering, авторами были выявлены следующие возможные направления атаки.
\bigskip

\subsection{Вызов ошибки в программе}
% TODO: заменить ссылки на главы
Во-первых, злоумышленник может вызвать возникновение ошибки в коде клиентской стороны. Самый простой пример - это замена указателя на стандартную функцию, предоставляемую интерфейсом браузера, указателем на HTML элемент (пример 2 из главы Постановка задачи). Последствиями такой атаки является некорректная дальнейшая работа веб-приложения.

% \subsection{Сброс пользовательских настроек}
% Во втором направлении атаки злоумышленник также подменяет важные для работы приложения объекты, но на этот раз не с целью вызова ошибки в коде, а для того, чтобы испортить пользовательские настройки. Такой тип атаки особенно актуален для описанных выше приложений вида Single Page Application, в которых обычно результаты работы пользователя сохраняются в некий объект (назовем его state) для последующей отправки на сервер.

\subsection{Обход критических состояний}
% TODO: заменить ссылки на главы
Для описания второго направления, проанализируем пример 3 из главы Постановка задачи:

\begin{lstlisting}
	<form name="is_in_black_list">
	<script>
		if (user_in_black_list()) { // точка 1
			is_in_black_list = true; // точка 2
		}
		if (is_in_black_list) { // точка 3
			forbid_action(); // точка 4
		}
	</script>
\end{lstlisting}
\bigskip

Предположим, что функция user\_in\_black\_list осуществляет посылку запроса на веб-сервер с целью выяснения находится ли пользователь в черном списке.


%TODO может быть вынести эти определения наверх?
Введем пару определений:
\begin{itemize}
	\item \textit{Состояние программы в точке $i$} - это тройка $<i, Vars, Values>$, где $i$ - точка программы, $Vars$ - множество переменных, доступных в $i$, $Values$ - множество двоек $<var, value>$, где $var \in Vars$, а $value$ - значение $var$ в $i$.  
	%TODO норм ли? критический - 2. Опасный, связанный с возможностью нарушения нормального состояния чего-нибудь
	\item \textit{Критическое состояние} - это такое состояние, в которое может перейти страница только при условии выполнения определенных требований на серверной стороне приложения.
	\item \textit{Некритическое состояние} - это состояние, не являющееся критическим.
	\item \textit{Допустимое для точки $i$ состояние} - это состояние $<i, Vars, Values>$, в котором программа может оказаться.
\end{itemize}


Например, состояние $<4, is\_in\_black\_list , <is\_in\_black\_list, true>>$ является критическим, так как для того, чтобы программа перешла в него, необходимо присутствие пользователя в черном списке.

\bigskip
С другой стороны, состояние $<4, is\_in\_black\_list, <is\_in\_black\_list, HTMLForm>>$  также является допустимым для точки 4, но не является критическим.
\bigskip

Другими словами, для точки 4 существуют два допустимых состояния, одно из которых дает нарушителю возможность нанести вред пользователю, который не находится в черном списке. Очевидно, такая ситуация является угрозой безопасности приложения.


С помощью введенных выше определений второе направление атаки можно описать так: \textit{С помощью недостатка DOM Clobbering нарушителям удается перевести программу в некритическое состояние в некоторой точке, для которой все прочие допустимые состояния являются критическими}

\subsection{Выполнение произвольного кода}
% TODO: заменить ссылки на главы
В-третьих, злоумышленники могут добиться исполнения произвольного кода в веб-браузере жертвы. Реальным примером такой атаки является пример 4 из главы Постановка задачи. Стоит отметить, что в данном случае методы эксплуатации идентичны методам эксплуатации недостатка DOM Based XSS. Данная атака является самой опасной из приведенных трех, так как она позволяет нарушителям осуществлять любые действия от имени жертвы.

\section{Требования к решению}
На основе проведенного исследования возможных направлений атаки были сформулированны следующие требование к алгоритму анализа веб-приложений.

\begin{enumerate}
	\item Алгоритм должен уметь детектировать возможность возникновения ошибок из-за переименования HTML элемента и места таких ошибок.
	%TODO уточнить формулировку
	\item Алгоритм должен уметь детектировать возможность и пути обхода критических состояний.
	\item Алгоритм должен уметь детектировать возможность выполнения произвольного кода.
\end{enumerate}

\section{Методика решения задачи}
Перед рассмотрением методики решения задачи, разработанной авторами, введем следующие определения:

%TODO: вставить ссылку на определения

Будем говорить, что \textbf{оператор О зависит от переменной A по данным}, если выполняется хотя бы одно из двух:
\begin{itemize}
	\item Оператор О использует значение А.
	\item Оператор О зависит по управлению от оператора, который зависит от А.
\end{itemize}


Будем говорить, что \textbf{переменная А зависит от переменной B по данным}, если выполняется хотя бы одно из двух:
\begin{itemize}
	\item Оператор, вычисляющий А, зависит от B.
	\item Оператор, вычисляющий А, зависит от переменной, которая зависит от B.
\end{itemize}

\bigskip
Приведем пару примеров:
\begin{lstlisting}
	a = "a";
	b = a + "b"; // b зависит от a
	c = b + "c"; // c зависит от b, а значит и от a
	if (a === "a") { 
		d = "d"; // d зависит от a
	}
	else {
		e = "e"; // e зависит от a
	}
	func(a); // оператор вызова функции зависит от a
\end{lstlisting}
\bigskip


Теперь всё готово для описания этапов методики.
\begin{enumerate}
	\item \textit{Этап подготовки} - для заданной веб-страницы с недостатком DOM Clobbering и заданного HTML элемента, имя которого можно изменять, выбирается очередное имя из заранее подготовленного списка имен.
	\item \textit{Этап анализа кода} - анализируется поток данных кода клиентской стороны приложения с целью построения списка всех переменных и операторов, зависимых по данным от рассматриваемого HTML элемента.
	\item \textit{Этап обработки результатов анализа} - по полученному списку зависимых переменных и операторов вычислить значение некоторых предикатов.
	\item \textit{Этап вывода} - результатом является список всех зависимых переменных и вычисленные значения предикатов.
\end{enumerate}
\bigskip

\section{Доказательство корректности методики}
Докажем, что полученная методика удовлетворяет всем требованиям, сформулированным выше.


Во-первых, рассмотрим пример с атакой путем вызова ошибки. Допустим, что в заранее подготовленном списке слов есть все имена HTML элемента, для которых в программе может произойти ошибка. Тогда, процесс исследования тривиален: 
\begin{enumerate}
	\item Выбрать следующее имя HTML элемента из списка имен.
	\item Запустить программу и проверить произошла ли ошибка.
	\item Если в списке имен еще остались имена, вернуться к шагу 1.
\end{enumerate}
То есть, корректность решения зависит от полноты списка имен. Авторами были исследованы и выписаны все имена из стандартного пространства имен объекта document, которые уязвимы к недостатку DOM Clobbering. Эти имена будут присутствовать в DOM любой веб-страницы и представляют собой особую опасность. Для нестандартных имен в представленной ниже реализации существует функционал расширения встроенного списка.

\bigskip
Для проверки возможности проведения атаки второго рода, достаточно проверить зависит ли хотя бы одна переменная или хотя бы один оператор, соответствующие критическим участкам кода, от выбранного HTML элемента. Если такие зависимости обнаружены, то в данных участках кода будут возможны некритические состояния, что, как было показано выше, ведет к угрозе безопасности.

\bigskip
%TODO ссылка на исследование с DOM Based XSS
Как уже отмечалось выше эксплуатация недостатка DOM Clobbering с выполнением произвольного кода идентична эксплуатации недостатка DOM Based XSS. Для данного недостатка исследователями был получен следующий результат: проведение атак возможно если пользовательский ввод попадает в так называемые стоки (особые переменные и вызовы особых функции) без должной обработки. В терминах, введенных нами, это утверждение можно переформулировать следующим образом: \textit{выполнение произвольного кода путем эксплуатации недостатка DOM Clobbering возможно в том случае, если хотя бы один сток зависит по данным от выбранного HTML элемента и данные попадают в этот сток без должной обработки}.


То есть, для ответа на вопрос о возможности проведения атак третьего вида, нужно ответить на двух других вопроса:
\begin{itemize}
	\item Существует ли зависимый сток?
	\item Если да, то в каком виде в него попадают данные?
\end{itemize}


С помощью предложенной методики можно получить ответы на оба вопроса:
\begin{itemize}
	\item Нужно проверить попал ли хотя бы один сток в список зависимых переменных и операторов.
	\item Нужно проверить какая обработка проводилась на пути от HTML элемента к стоку в графе потока данных.
\end{itemize}
\bigskip

На этом доказательство соответствия предложенной методики всем необходимым требованиям завершено.

